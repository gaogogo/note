\documentclass[a4paper,UTF8]{article}
\usepackage{ctex}
\usepackage[margin=1.25in]{geometry}
\usepackage{amsmath}
\usepackage{listings}

\begin{document}
\section*{问题求解作业}
\subsection{从P到NP}
\textbf{1 Ask the oracle} Suppose that we have access to an oracle, as in Promble 1.10,
who can answer thedecision problem SAT. show that, by asking her a ploynomial
number of questions,we can find a satisfying assignment if one exists. Hint:
first describe how setting a variable gives a smaller SAT formula on the remaining
variable. Show that this is ture of SUBSET SUM as well. This property is called
self-reducibility, and we will meet it again in chapter 9 and 13.

 \vspace{6pt}

\textbf{2 Same-sum subset} Suppose that $S$ is a list of 10 distinct integers, each
ranging from 0 to 100. Show that there are two distinct, disjoint, sublist $A,B\subset S$
with the same total. (Note that, unlike in INTEGER PARTITIONING, we don't demand that $A\cup B = S$).
That doesn't make it easy to find one!

 \vspace{6pt}

\textbf{3 When greed works} A sequence $a_{1},...,a_{n}$ of integers is called superincreasing
if each element $a_{i}$ is strictly greater than the sum of all previous elements.
Show that SUBSET SUM can be solved in ploynomial time if $S$ is superincreasing.
Hint: be greedy. What does your algorithm do in the case $S=\{1,2,4,8,...\}$.

\vspace{6pt}

\textbf{From circuits to NAESAT} Find a direct reduction from CIRCUIT SAT to NAE-3-SAT.
Hint: consider the 3-SAT clause $(\bar{x}_{1} \vee \bar{x}_{2} \vee \bar{y})$ in our description
of an AND gate. Does it ever happen that all three literals are ture?




\end{document}
