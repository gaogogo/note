\documentclass[a4paper,UTF8]{article}
\usepackage{ctex}
\usepackage[margin=1.25in]{geometry}
\usepackage{amsmath}
\usepackage{listings}

\begin{document}
\section*{问题求解作业}
\subsection{从P到NP}
\textbf{1 Ask the oracle} Suppose that we have access to an oracle, as in Promble 1.10,
who can answer thedecision problem SAT. show that, by asking her a polynomial
number of questions,we can find a satisfying assignment if one exists. Hint:
first describe how setting a variable gives a smaller SAT formula on the remaining
variable. Show that this is ture of SUBSET SUM as well. This property is called
self-reducibility, and we will meet it again in chapter 9 and 13.

 \vspace{6pt}

\textbf{2 Same-sum subset} Suppose that $S$ is a list of 10 distinct integers, each
ranging from 0 to 100. Show that there are two distinct, disjoint, sublist $A,B\subset S$
with the same total. (Note that, unlike in INTEGER PARTITIONING, we don't demand that $A\cup B = S$).
That doesn't make it easy to find one!

 \vspace{6pt}

\textbf{3 When greed works} A sequence $a_{1},...,a_{n}$ of integers is called superincreasing
if each element $a_{i}$ is strictly greater than the sum of all previous elements.
Show that SUBSET SUM can be solved in polynomial time if $S$ is superincreasing.
Hint: be greedy. What does your algorithm do in the case $S=\{1,2,4,8,...\}$.

\vspace{6pt}

\textbf{4 From circuits to NAESAT} Find a direct reduction from CIRCUIT SAT to NAE-3-SAT.
Hint: consider the 3-SAT clause $(\bar{x}_{1} \vee \bar{x}_{2} \vee \bar{y})$ in our description
of an AND gate. Does it ever happen that all three literals are true?

\vspace{6pt}

\textbf{5 Easy majority?} Consider MAJORITY-OF-3SAT, in which each clause contains
three literals and at least two of them must be true for it to be satisfied. Either
show that this problem is in P by reducing it to 2-SAT, or show that it is NP-complete
by reducing 3-SAT to it.

\subsection{P=NP?}
\textbf{1 This far and no farther} Consider the version of INDEPENDENT SET given at
end of section 4.2.4:
\begin{center}
    \fbox{\shortstack[l]{INDEPENDENT SET (exact version) \\ Input: A graph $G=(V,E)$
    and an integer $k$ \\ Question: Does $G$'s largest independent set have size $k$?}}
\end{center}
Or similarly, the problem CHROMATIC NUMBER from Problem 4.22:
\begin{center}
    \fbox{\shortstack[l]{CHROMATIC NUMBER \\ Input: A graph $G$
    and an integer $k$ \\ Question: Is $k$ the smallest integer such that $G$ is $k$-colorable?}}
\end{center}
show that for both these problems, the property $A(x)$ that $x$ is a yes-instance can be expressed
as follows, where $B$ and $C$ are in P:
$$ A(x) = (\exists y : B(x, y)) \wedge (\forall z : C(x, z)).$$
Equivalently, show that there are two properties $A_{1}$, $A_{2}$ in NP such that $A(x)$
holds exactly if $A_{1}(x)$ is true but $A_{2}(x)$ is false. Thus $A$ can be thought
of as the difference between two NP properties, and the class of such properties
is called DP. Now show that $$ DP \subseteq \sum_{2} P \cap  \prod_{2} P $$
and give same intuition for why this inclusion should be proper.

\vspace{6pt}

\textbf{2 Mate in $k$} Argue that "mate in $k$ moves" problems in Generalized Chess
are in $\sum_{k} P$ even on a borad of exponential size, as long as there are only
a polynomial number of pieces. Hint: you just need to check that the moves can be
described with a polynomial number of bits, and that we can check in polynomial
time, given the position and the sequence of moves, whether White has won.

\vspace{6pt}

\textbf{3 Constructible time} Prove tha the function $f(n) = 2^{n}$ is time constructible.
Prove also that if $f(n)$ and $g(n)$ are time constructible, then so are $f(n) + g(n)$,
$f(n) \times g(n)$, and $f(g(n))$, where for the last one we assume that $f(n) \geq n$.

\vspace{6pt}

\textbf{4 the oracle says yes or on} Let's explore the class $P^{NP}$ in more detall.
First, show that $P^{NP}$ contains both NP and coNP. then, show that
$P^{NP} \subseteq \sum_{2} P \cap  \prod_{2} P$ and also that $NP^{NP}= \prod_{2}P$.
Hint: let $\prod$ be the polynomial-time program that makes calls to the NP oracle.
An input $x$ is a yes-instance if there is a run of $\prod$ which returns "yes"
which receives "yes" and "no" answers from the oracle for some NP-complete.

\subsection{解难题新思路}
\textbf{1} Construct, for any positive integer $n$, a graph $G_{n}$, such that the
optimal vertex cover has the cardinality $n$ and the algorithm VCA can computer a vertex
cover of the cardinality $2n$.

\vspace{6pt}

\textbf{2} Find, for any positive integer $n \geq 3$, a weight function $c_{n}$
for the complete graph $K_{n}$ of $n$ vertices, such that there exists at least two
different weights on the edges of $K_{n}$ and the algorithm SB always computes an
optimal solution.

\vspace{6pt}

\textbf{3} Let $H$ be Hamiltonian tour in a graph $G$. Remove three edges $\{ a, b \}$,
$\{c, d\}$ and $\{e, f\}$ such that $\left |  \{ a, b, c, d, e, f\}\right | = 6$ and
$H$ visits these 6 vertices in the order $a, b, c, d, e, f$. Draw all possible triples
of edges whose addition to $H$ results again in a Hamiltonian tour. How many possibilities
are there, if $k$ edges foming a matching are removed for $k \geq 3$?

\vspace{6pt}
\textbf{4} Prove that LS-CUT is a polynomial-time algorithm.

\subsection{应用视角下的树结构}
\textbf{1} 假设输入一个所有元素均不同的序列,其元素取自全序集,设计算法找出序列中第二大的
元素,并分析其时间代价。

\vspace{6pt}

\textbf{2} Describe a modified version of the B-tree instance algorithm so that
each time we creat an overflow because of a split of a node $v$, we redistribute
keys among all of $v$'s siblings, so that each siblings holds roughly the same number
of keys (possibly cascading the split up to the parent of $v$).

\vspace{6pt}
\textbf{3} 写一个算法,在 R-树中加入一个矩形,可能需要分隔节点。讨论预期的代价。

\subsection{网络流经典算法拓展}
\subsection{Steiner树-限制解空间}
\subsection{调度问题}
\subsection{结果质量有保证的近似算法}
\subsection{ }
\subsection{随机算法的基本概念}
\subsection{随机舍入}
\textbf{1} Prove that there is also a deterministic 0.5-approximation algorithm for
the MAXCUT problem.

\vspace{6pt}

\textbf{2} Prove that there is 0.5(1+1/$m$)-approximation algorithm (randomized or
deterministic) for the MAXCUT problem, where $m$ is the number of edges of the given
graph $G$.

\subsection{bin packing问题}
\subsection{受限最短路径问题}
\subsection{几何搜索问题}
\subsection{property testing}
\end{document}
